\documentclass[10pt,reqno]{amsart}
\usepackage{amsmath,times,hyperref}
\usepackage{amsthm, amssymb, amsfonts, mathrsfs, float, booktabs, subfigure, boxedminipage, bbold}
\usepackage{latexsym}
\usepackage{graphicx}
%%  Renewed
\renewcommand{\phi}{\varphi}
\renewcommand{\epsilon}{\varepsilon}
\renewcommand{\Re}[1]{\operatorname{Re} #1}
\renewcommand{\Im}[1]{\operatorname{Im} #1}
%\renewcommand{\sech}{\operatorname{sech}}
\linespread{1.2}

%% New Commands
\newcommand{\dD}{\partial \mathbb{D}}
\newcommand{\Z}{\mathbb{Z}}
\newcommand{\R}{\mathbb{R}}
\newcommand{\Q}{\mathbb{Q}}
\newcommand{\C}{\mathbb{C}}
\newcommand{\K}{\mathbb{K}}
\renewcommand{\P}{\mathbb{P}}
\newcommand{\N}{\mathbb{N}}
\newcommand{\cl}{\operatorname{cl}}
\newcommand{\ran}{\operatorname{ran}}
\newcommand{\norm}[1]{\| #1 \|}
\newcommand{\inner}[1]{\left< #1 \right>}
\newcommand{\blf}{ {[\,\cdot\, , \,\cdot\,]} }
\newcommand{\h}{\mathcal{H}}
\newcommand{\M}{\mathcal{M}}
\newcommand{\E}{\mathcal{E}}
\newcommand{\V}{\mathcal{V}}
\newcommand{\W}{\mathcal{W}}
\newcommand{\T}{\mathbb{T}}
\newcommand{\dom}{\mathcal{D}}
\newcommand{\pc}{\perp_C}
\newcommand{\vecspan}{\operatorname{span}}
\newcommand{\interior}{\operatorname{int}}
\newcommand{\lcm}{\operatorname{lcm}}
\newcommand{\tr}{\operatorname{tr}}
%%  Matrices
\newcommand{\minimatrix}[4]{\begin{pmatrix} #1 & #2 \\ #3 & #4 \end{pmatrix}  }
\newcommand{\megamatrix}[9]{\begin{pmatrix} #1 & #2 & #3 \\ #4 & #5 & #6 \\ #7 & #8 & #9\end{pmatrix}  }

\renewcommand{\labelenumi}{(\roman{enumi})}

\newcommand{\twovector}[2]{\begin{pmatrix} #1\\#2 \end{pmatrix} }
\newcommand{\threevector}[3]{\begin{pmatrix} #1\\#2\\#3 \end{pmatrix} }

\renewcommand{\vec}[1]{{\bf #1}}

\newcommand{\varnot}{\sim\!\!}
\newcommand{\due}[1]{\vspace{-0.2in}\begin{center}\textsc{due at the beginning of class \underline{#1}} \end{center}\medskip }


%%%
%%% Theorem Styles
%%%
\newtheorem{Result}{Result}
\newtheorem{Proposition}{Proposition}
\newtheorem{Corollary}{Corollary}
\newtheorem{Theorem}{Theorem}
\newtheorem*{Thm}{Theorem}
\newtheorem{Postulate}{Postulate}
\newtheorem{Lemma}{Lemma}
\theoremstyle{definition}
\newtheorem*{Definition}{Definition}
\newtheorem*{Example}{Example}
\newtheorem*{Remark}{Remark}
\newtheorem{Exercise}{Exercise}
\newtheorem*{Question}{Question}


\allowdisplaybreaks
%\newcounter{ex}[section]



%\numberwithin{section}{chapter}
%\numberwithin{Theorem}{chapter}
\numberwithin{equation}{section}
%\numberwithin{Example}{chapter}

%%
%% MAIN DOCUMENT
%%
\begin{document}
\title{Machine learning on $k$-set of $n$ sets}
\author{Tum Chaturapruek, Jeremy Usatine, 
\\Professor Susan Martonosi, Professor Michael Orrison} 
\maketitle

\begin{abstract}
We use similar techniques from Huang's thesis to 
work on our problem. The hard part
is to translate between them.
\end{abstract}


%------------------------------------------------------ 
\section{Introduction}
We first review Jonathan Huang's thesis.
\subsection{From Chapters 1-4}
\begin{enumerate}
\item  distribution on the symmetric group:
mutually exclusivity constraints
\[h(\sigma(i) = \sigma(j)) = 0,\]
whenever $i\neq j.$
\item Hidden Markov Model (HMM):
\[h(\sigma^{(1), \ldots, \sigma^{(T)}, 
z^{(1)}, \ldots, z^{(T)}})
=
h(\sigma^{(1)}) h(z^{(1)}|\sigma^{(1)})
\prod_{t=2}^T h(z^t|\sigma^{(t)}) h(\sigma^{(t)}|\sigma^{t-1}),
\]
where the $\sigma^{(i)}$ are latent permutations, and
the $z^{(i)}$ are observed variables.
\end{enumerate}
\subsection{From Chapter 5}
We begin with the following theorem.

\noindent\textbf{Definition:} A \textit{representation} of a group $G$
is a map $\rho$ from $G$ to a set of
invertible $d_\rho\times d_\rho$ (complex)
matrix operators ($\rho:G\to \C^{d_\rho\times d_\rho}$)
which preserves algebraic structure.

We consider three types of representations in this paper
\begin{enumerate}
\item trivial representation (maps everything to 
1). Although trivial, it is still important.
\item The first-order permutation representation:
$[\tau_{(n-1, 1)(\sigma)}]_{ij}
= \mathbb{1}\{\sigma(j) = i\}
$
\item The alteration representation
\[\rho_{(1,1,1, \ldots, 1)}\]
which maps to $\pm 1$ based on the parity of
the permutation.
\end{enumerate}

\noindent \textbf{Theorem} (James Submodule Theorem).
The marginals corresponding to ``simple'' partition
will require very few Fouries coefficients to compute.
\begin{enumerate}
\item  First-order marginal probability:
$\lambda = (n-1, 1).$
\item  Second-order marginal probability:
$\lambda = (n-2, 2).$
\end{enumerate}

\begin{enumerate}
\item (uniqueness) For each partition $\lambda$ of $n$,
there exists an irreducible representation
$\rho_\lambda$ which is unique.
\item (completeness) Every irreducible representation
of $S_n$ corresponds to some partition of $n$.
\item There exists a matrix $C_\lambda$ 
associates with each partition $\lambda$ for which
\[C_{\lambda}^T \tau_\lambda (\sigma) C_\lambda
=
\oplus_{\mu \unrhd \lambda}\oplus_{l=1}^{k_{\lambda,\mu}} \rho_{\mu}(\sigma)\]
for all $\sigma\in S_n.$
\end{enumerate}


\noindent \textbf{Theorem} (Peter-Weyl Theorem).
The matrix entries of the irreducibles
of $G$ form a complete set of orthogonal
basis functions on $G$.


\noindent \textbf{Definition} (Fourier Transform).
For the probability distribution $h:G\to \R$
\[\hat{h}_\rho = \sum_\sigma h(\sigma)\rho(\sigma).\]

\noindent \textbf{Theorem} (Fourier Inversion Theorem).
We have
\[h(\sigma) = \frac{1}{|G|}\sum_{\lambda}
d_{\rho_\lambda} \text{Tr}[\hat{h}^T\cdot \rho_\lambda(\sigma)],\]
where $\lambda$ indexes over the collection
of irreducibles of $G$.

\textit{Remark}: The trace can be though of as the inverse
if the representations are unitary, 
i.e., $\rho(\sigma)^{-1} = \rho(\sigma)^T.$ To
see this in more details,
let 
\[A =
\begin{bmatrix}
a & b\\
c & d
\end{bmatrix},
B =
\begin{bmatrix}
e & f\\
g & h
\end{bmatrix}.
\]
We have $A\,.*B = ae + bf + cg + dh = \text{Tr}(AB^T).$

\noindent \textbf{Definition} (Dominance Ordering).
Let $\lambda, \mu$ be partitions of $n$. Then 
$λ \unrhd µ$ (we say $\lambda$ dominates $\mu$),
if for each $i$, 
\[\sum_{k=1}^i \lambda_k \geq \sum_{k=1}^i \mu_k.\]
For example, $(4,2)\unrhd (3,2,1).$
Note that high dominance ordering 
corresponds to low frequency, and vice versa.
%------------------------------------------------------ 
\subsection{From Chapter 6 (Tableaux Combinatorics
and Algorithms)}
\begin{enumerate}
\item  The dimension of the irreducible 
representation $\rho_\lambda$ can be 
indexed by a collection of the 
standard Young tableaux.
\item Clausen's FFT algorithm.
\item The Gel'fand-Tsetlin basis.
\end{enumerate}


\section{Our thoughts}
We have
\begin{equation}
V = S^{(n)}\bigoplus S^{(n-1, 1)} \bigoplus
\cdots \oplus S^{(n-k,k)}.
\end{equation}
Each space can be thought of pure $i$-th
order effect.

Idea: embed out space $V$
into a bigger space $S_n$.
Then break it down to irreducibles.
Then pull it back.

Some keywords/more ideas
\begin{enumerate}
\item spherical groups
\item replace $x$ by $G_x$
\item young ? homogeneous groups
\item Gelfand pair
\item Clebsch-Gordan series.
\end{enumerate}

From part 8, we note that,
a Kronecker (Tensor) Product Representation.
If $\rho_\lambda$ and $\rho_\mu$ are any
two irreducibles of $G$, there exists a similarity
transform $C_{\lambda\mu}$ (Clebsch-Gordan coefficients) such that, for any
$\sigma\in G$
\[C_{\lambda\mu}^{-1}\cdot[\rho_\lambda\otimes\rho_mu]
\cdot C_{\lambda\mu}
=
\bigoplus_{\nu}\bigoplus_{l=1}^{z_{\lambda\mu\nu}}
\rho_\nu(\sigma),
\]
where $z_{\lambda\mu\nu}$ is known 
as Clebsch-Gordan series.

\subsection{Research Meeting on November 1, 2012}
Some notes from Huang: 
\begin{enumerate}
\item Proposition 11. Page 29 in pdf 
of Huang 2009 (about 
$[\hat{fg}]_{\rho_\nu}$
in terms of the sum of the tensor products
of $\hat{f}_{\rho_\lambda}\otimes \hat{g}_{\rho_\mu}$.

\item Theorem 30 (Murnaghan's formulas)  (page 65 in pdf) Let $\rho_1,\rho_2$ be the irreducibles corresponding to the partition 
$(n - p, \lambda_2 ,...)$ and
$(n - q, \mu_2 ,...)$ respectively. Then the product
$\rho_1\otimes \rho_2$ does not contain any irreducibles corresponding to a partition whose first term is less than $n - p - q$.
\end{enumerate}

Let $G = S_n$ and $H = \{1,2,3,\ldots, k\}.$
We can think $\C[G/H]$ and $\C[G]$
as $\C G$-modules.

Suppose $\overline{g}, \overline{f}\in \C[G/H],$
$\phi:\overline{f}\mapsto g\in \C[G]$ defined by
\[\phi(\overline{f})(x) = \overline{f}(xH)\]

The inverse map is $\psi:f\to \C[G/H]$ defined by
\[\psi(f)(xH) = \frac{1}{|H|}\sum_{y\in H}f(y)\]

\noindent \textbf{Theorem} (Shur's Lemma).
Let $T:U\to U'$ be a $\C G$-module homormorphism,
and suppose $U,U'$ are irreducibles. Then,
\[T=
0 \text{ if } U\not\cong U',
\]
or $T$ is an isomorphism if $T\neq 0.$
In our case, we have
\begin{equation}
\label{map}
\C[G/H]\xrightarrow{\phi}\C[G]\xrightarrow{\text{DFT}}
\hat{\C[G]}.
\end{equation}
Then, to pull back, we need
\[\hat{\C[G]}\xrightarrow{\text{Theorems, DFT}^{-1}}\C[G]\xrightarrow{\psi}
\hat{\C[G/H]}.\]
Thus, the main idea here is to never leave the Fourier
space.

\subsection{Some notes prepared 
for research meeting on November 8, 2012}
If $f\in \C[G]$ is a $j$-th order observation on
the nonzero Fourier coefficients correspond to
irreducibles indexed by 
\[(n-r, r)\]
with $r\leq j.$

Def: $f$ is a $j$-th order probability distribution
if it is fully defined by unordered $j$-th order marginal distribution.

1. Suppose $f, g\in \C[S_n]$ and $\hat{f}, \hat{g}$ are the FTs.

If $[\hat{f}]_{\rho_r}$ is zero for all
$\lambda\neq (n-r, r)$ for all $r\leq j$
and
$[\hat{g}]_{\rho_r}$ is zero for all
$\lambda\neq (n-r, r)$ for all $r\leq i$
then
$[\hat{fg}]_{\rho_r}$ is zero for all
$\lambda\neq (n-r, r)$ for all $r\leq j + i$.

2. Suppose $u,v \in \C$[$k$-set of an $n$-set],
$u$ is the $i$-th order, v is the $j$-th order.
Then,
\[uv = fg \{1, 2, 3, \ldots, k\}\]
for $f$ and $g$ as above.


\subsection{Meeting on November 8, 2012}
\begin{enumerate}
\item Gelfand pair
\item Harmonic analysis. Double cosets:
\[KgK = \bigcup_{k\in K} (kg)K,\]
also denoted as $S_2\Sigma_3/S_2.$
\item When someone mentions something like
a spherical Fourier transform, think
of the computational time. Is there
an analogy like FT and FFT?
\item $S_K\times S_{n-k}$ multiplicities free.
\item The operation $\varphi$ and 
DFT in the Diagram.~\eqref{map} have
to work together to get
\[\hat{\C[G]}\cong
A_1\oplus A_2\oplus A_3\cdots \oplus A_n,\]
where $A_i$ only has the first column
nonzero. Each column is irreducible representation
of the group algebra.
\item Gelfand pair. Embed in a clever way.
Can embed on $\C[G/H]$ to our group.
\item spherical functions via random walks
on groups.
\end{enumerate}

%\nocite{*}
%\bibliographystyle{plain}
%\bibliography{bib1}
\end{document}
